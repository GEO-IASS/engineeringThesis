\chapter{Propuesta} % Main chapter title
\label{Capitulo3} % Change X to a consecutive number; for referencing this chapter elsewhere, use \ref{ChapterX}
\lhead{\emph{Propuesta}}

%----------------------------------------------------------------------------------------
%	SECTION 1
%----------------------------------------------------------------------------------------
%algorithm uses Daubechies wavelet for spatial dimension and Fourier transform for vertical dimension. 
\section{Proceso}
Como se pudo capitalizar en el marco teórico del documento, los diversos pasos para llegar al resultado esperado constan de mejoras que se realizarían después de la etapa de obtención de los las imagenes hiperespectrales realizadas por la camara con estructura CTIS. Los métodos (o por decir de otra forma, filtros) son independientes entre sí, por lo que uno de ellos no necesita de la ejecución de otro previo para poderse aplicar. 
La propuesta se basa en utilizar uno, o combinación de los siguientes métodos encontrando así el mejor camino hacia el resultado esperado:

\begin{itemize}
\item Daubichies.
\item Slices.
\item Mosaic-Malvar.
\item Blur-Shparpender.
\end{itemize}

Donde cada uno de dichas combinaciones será probada por la calidad percibida mediante el ojo humano. 
Para poder analizar las imágenes se establecieron rutas de resultados que se muestran en la Figura \ref{pics:squeme}, donde se utilizaron diversas combinaciones de la teoría mostrada en los capítulos anteriores.
\clearpage
\begin{figure}[h]
\includegraphics[scale=.4]{./images/RESULTS/squeme.jpg}
\caption{Esquema de resultados.}
\label{pics:squeme}
\end{figure}
Cada cuadro corresponde a una imagen superposicionadas y el conjunto de imágenes a las obtenidas por el procedimiento de abstracción por longitud de onda electromagnética de la imagen suporposicionada. 