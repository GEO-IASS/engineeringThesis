\chapter{Glosario} % Main appendix title
\lhead{\emph{Glosario}}

\textbf{Interpolación Bilineal}

La interpolación consiste en hallar un dato dentro de un intervalo en el que conocemos los valores en los extremos. La interpolación se dirá lineal cuando sólo se tomen dos puntos y cuadrática cuando se tomen tres.

\textbf{Interpolación Cromática}

Un algoritmo de interpolación cromática (conocida en inglés como demosaicing o demosaicking) es un proceso digital de imagen utilizado para reconstruir una imagen en color mediante las muestras cromáticas incompletas adquiridas desde un sensor de imagen recubierto con un mosaico de filtro de color.

\textbf{Mosaico de Bayer}

El filtro, máscara o mosaico de Bayer es un tipo de matriz de filtros, rojos verdes y azules, que se sitúa sobre un sensor digital de imagen para hacer llegar a cada fotodiodo la información de luminosidad correspondiente a una sección de los distintos colores primarios.

\textbf{Laplaciano}

 Es un operador diferencial elíptico de segundo orden, denotado como Δ, relacionado con ciertos problemas de minimización de ciertas magnitudes sobre un cierto dominio.

\textbf{Colimación}

Es un sistema que a partir de un haz (de luz, de electrones, etc.) divergente obtiene un "haz" paralelo. Sirve para homogeneizar las trayectorias o rayos que, emitidos por una fuente, salen en todas direcciones y obtiene un chorro de partículas o conjunto de rayos con las mismas propiedades.

\textbf{RGB}

Mosaico de Bayer con patrón de secuencia Rojo, Verde y Azul.

\textbf{RGGB}

Mosaico de Bayer con patrón de secuencia Rojo, Verde, Verde y Azul.

\textbf{slit}

Es una salida de sensor bidimensional (2-D) que representa un espectro de rendija completa (x, λ).

\textbf{snapshot}

Consiste en un único tiempo de integración gracias a un conjunto de detectores.
